% Created 2013-04-05 Fri 10:42
\documentclass[english]{rcalibrionecolumn}


\providecommand{\alert}[1]{\textbf{#1}}

\title{\color{statblue}{Emacs Shortcuts}}
\author{\color{statblue}Ronert Obst}
\date{\color{statblue}\today}
\hypersetup{
  pdfkeywords={},
  pdfsubject={},
  pdfcreator={Emacs Org-mode version 7.9.3f}}

\begin{document}

\maketitle

\setcounter{tocdepth}{3}
\tableofcontents
\vspace*{1cm}
\section{Help}
\label{sec-1}


\begin{center}
\begin{tabular}{ll}
 \textbf{Action}  &  \textbf{Shortcut}  \\
\hline
 help-command     &  <f1>               \\
 apropos          &  <f1>-a             \\
 describe-mode    &  <f1>-m             \\
\end{tabular}
\end{center}
\section{Buffers, Windows, Frames and Files}
\label{sec-2}


\begin{center}
\begin{tabular}{ll}
 \textbf{Action}                    &  \textbf{Shortcut}  \\
\hline
 find-name-dired                    &  C-c s              \\
 dired-jump                         &  C-x C-j            \\
 find recent file                   &  C-x f              \\
 rename-current-buffer-file         &  C-x w              \\
 delete-current-buffer-file         &  C-x C-k            \\
 In ido-find-file go to home        &  C-c h              \\
\hline
 ibuffer                            &  C-x C-b            \\
\hline
 rotate-windows                     &  C-x -              \\
 find-file-other-window             &  C-x M-f            \\
\hline
 create new frame                   &  C-x C-n            \\
\hline
 winner-undo                        &  C-c <up>           \\
 winner-redo                        &  C-c <down>         \\
\hline
 split-window-horizontally-instead  &  C-x vertical line  \\
 split-window-vertically-instead    &  C-x underscore     \\
\hline
 window-jump-down                   &  C-2                \\
 window-jump-up                     &  C-8                \\
 window-jump-left                   &  C-4                \\
 window-jump-right                  &  C-6                \\
\hline
 switch-space                       &  C-5                \\
 new-space                          &  C-7                \\
 save-space                         &  C-9                \\
 kill-space                         &  C-1                \\
\end{tabular}
\end{center}
\section{Navigation}
\label{sec-3}


\begin{center}
\begin{tabular}{ll}
 \textbf{Action}                  &  \textbf{Shortcut}  \\
\hline
 ace-jump-line-mode               &  jj                 \\
 ace-jump-word-mode               &  hh                 \\
 jump-char-forward                &  kk                 \\
 jump-char-backward               &  aa                 \\
 iy-go-to-char                    &  öö                 \\
 beginning-of-defun               &  C-M-a              \\
 end-of-defun                     &  C-M-e              \\
\hline
 ergoemacs-forward-block          &  S-<down>           \\
 ergoemacs-backward-block         &  S-<up>             \\
 ergoemacs-backward-open-bracket  &  S-<left>           \\
 ergoemacs-forward-close-bracket  &  S-<right>          \\
\hline
 es-move-text-right               &  C-S-<right>        \\
 es-move-text-left                &  C-S-<left>         \\
 es-move-text-up                  &  C-S-<up>           \\
 es-move-text-down                &  C-S-<down>         \\
\hline
 move up 5 lines                  &  C-S-p              \\
 move down 5 lines                &  C-S-n              \\
 move right 5 characters          &  C-S-f              \\
 move left 5 characters           &  C-S-b              \\
\end{tabular}
\end{center}
\section{Bookmarks}
\label{sec-4}


\begin{center}
\begin{tabular}{ll}
 \textbf{Action}  &  \textbf{Shortcut}  \\
\hline
 bookmark-set     &  F13                \\
 bookmark-list    &  F14                \\
 bookmark-jump    &  F15                \\
\end{tabular}
\end{center}
\section{Search}
\label{sec-5}


\begin{center}
\begin{tabular}{ll}
 \textbf{Action}                &  \textbf{Shortcut}  \\
\hline
 multi-occur-in-this-mode       &  C-c r              \\
 rgrep                          &  C-x C-r            \\
 prelude-ido-goto-symbol        &  C-c i              \\
\hline
 ack-and-a-half                 &  C-c 1              \\
 ack-and-a-half-same            &  C-c 2              \\
 ack-and-a-half-find-file       &  C-c 3              \\
 ack-and-a-half-find-file-same  &  C-c 4              \\
\hline
 ag-regexp                      &  C-c 5              \\
 ag-project                     &  C-c 6              \\
 ag-project-at-point            &  C-c 7              \\
 ag-regexp-project-at-point     &  C-c 8              \\
\hline
\end{tabular}
\end{center}
\section{Projectile}
\label{sec-6}


\begin{center}
\begin{tabular}{ll}
 \textbf{Actions}                                    &  \textbf{Shortcut}  \\
\hline
 Display a list of all files in the project          &  C-c p f            \\
 Display a list of all test files                    &  C-c p T            \\
 Run grep on the files in the project                &  C-c p g            \\
 Display a list of all open project buffers          &  C-c p b            \\
 Runs `multi-occur` on all open project buffers      &  C-c p o            \\
 Runs interactive query-replace  files in  projects  &  C-c p r            \\
 Invalidates the project cache (if existing)         &  C-c p i            \\
 Regenerates the projects `TAGS` file                &  C-c p R            \\
 Kills all project buffers                           &  C-c p k            \\
 Opens the root of the project in `dired`            &  C-c p d            \\
 Shows a list of recently visited project files      &  C-c p e            \\
 Runs `ack` on the project                           &  C-c p a            \\
 Runs a standard compilation command                 &  C-c p l            \\
 Runs a standard test command                        &  C-c p p            \\
 Adds the currently visited to the cache             &  C-c p z            \\
 Display a list of known projects you can switch to  &  C-c p s            \\
\end{tabular}
\end{center}
\section{Helm}
\label{sec-7}


\begin{center}
\begin{tabular}{ll}
 \textbf{Action}  &  \textbf{Shortcut}  \\
\hline
 helm-mini        &  C-c h              \\
 helm-descbinds   &  C-c C-h            \\
 helm-projectile  &  C-c p h            \\
\end{tabular}
\end{center}
\section{Dired}
\label{sec-8}


\begin{center}
\begin{tabular}{ll}
 \textbf{Action}  &  \textbf{Shortcut}  \\
\hline
 wdired           &  F12                \\
\end{tabular}
\end{center}
\section{Version Control}
\label{sec-9}

\begin{longtable}{ll}

 \textbf{Action}                            &  \textbf{Shortcut} \\
\hline
\endhead
\hline\multicolumn{2}{r}{Continued on next page}\
\endfoot
\endlastfoot
 magit-init                                 &  F17                \\
 magit-status                               &  C-x g or F16       \\
 refresh status                             &  g                  \\
 \textbf{Sections}                          &                     \\
\hline
 toggle visibility of current section       &  TAB                \\
 toggle visib. of selec. and children       &  S-TAB              \\
 expand current sec. to detail level        &  1, 2, 3 and 4      \\
 expand all sec. to detail level            &  M-1, 2, 3, 4       \\
\hline
 \textbf{Untracked Files}                   &                     \\
 stage                                      &  s                  \\
 ignore file                                &  i                  \\
 prompt for file to ignore                  &  C-u i              \\
\hline
 \textbf{Staging and Commiting}             &                     \\
 stage current hunk                         &  s                  \\
 unstage current hunk                       &  u                  \\
 stage all hunks                            &  S                  \\
 unstage all hunks                          &  U                  \\
 discard uncomitted changes                 &  k                  \\
 prepare for commit                         &  c                  \\
 execute commit                             &  C-c C-c            \\
\hline
 \textbf{History}                           &                     \\
 history                                    &  l                  \\
 verbose history                            &  L                  \\
 inspect commit                             &  RET                \\
 copy sha1 of current commit to kill ring   &  C-w                \\
 show diff between current and marked com.  &                     \\
 mark current commit                        &  ..                 \\
 unmark current commit if marked            &  .                  \\
 magit toggle whitespace                    &  W                  \\
 grep history                               &  s                  \\
\hline
 \textbf{Diff}                              &                     \\
 shwo changes working tree and head         &  d                  \\
 show changes two arbitrary revisions       &  D                  \\
 apply current changes to working tree      &  a                  \\
\hline
 \textbf{Resetting}                         &                     \\
 reset current head to chosen revision      &  x                  \\
 reset working tree and staging area        &  X                  \\
\hline
 \textbf{Branching}                         &                     \\
 switch to different branch                 &  b                  \\
 create and switch to new branch            &  B                  \\
\hline
 \textbf{Pushing and Pulling}               &                     \\
 git push                                   &  P                  \\
 git push to specified remote repository    &  C-u P              \\
 git remote update                          &  f                  \\
 git pull                                   &  F                  \\
\hline
 toggle git-gutter                          &  F18                \\
 popup-diff git-gutter                      &  F19                \\
 vc-annotate                                &  C-x v g            \\
\end{longtable}
\section{Text Manipulation}
\label{sec-10}


\begin{center}
\begin{tabular}{ll}
 \textbf{Action}              &  \textbf{Shortcut}  \\
\hline
 hippie-expand                &  C-. or ..          \\
 auto-complete                &  C-, or ,,          \\
 yas-expand                   &  C- -               \\
\hline
 browse-kill-ring             &  C-x y              \\
 undo-tree-visualize          &  C-c v              \\
 query-replace-regexp         &  M-\&               \\
 cleanup-buffer               &  C-c ß              \\
 prelude-cleanup-buffer       &  M-ß                \\
 align-regexp                 &  C-x //             \\
 linum-mode                   &  C-<f6>             \\
\hline
 open-line-below              &  ii                 \\
 open-line-above              &  uu                 \\
 duplicate-line               &  C-c n              \\
 join-line                    &  C-x a              \\
 move-line-up                 &  M-S-up             \\
 move-line-down               &  M-S-down           \\
 kill-lines                   &  C-c C-<backspace>  \\
\hline
 zap-to-char                  &  üü                 \\
 zap-up-to-char               &  ää                 \\
 kill-back-to-indentation     &  C-M-<backspace>    \\
\hline
 comment-or-uncomment-region  &  C-c c              \\
 uncomment-region             &  C-c u              \\
\hline
 mark-whole-buffer            &  C-c m              \\
 mark-defun                   &  C-M-h              \\
 mc/mark-all-like-this        &  C-ä                \\
 mc/mark-previous-like-this   &  C-ü                \\
 mc/mark-next-like-this       &  C-ö                \\
 expand-region                &  - -                \\
 move with expand region      &  s-<arrow>          \\
\end{tabular}
\end{center}
\section{Macros}
\label{sec-11}


\begin{center}
\begin{tabular}{ll}
 \textbf{Action}           &  \textbf{Shortcut}  \\
\hline
 defining-kbd-macro        &  <f3>               \\
 kmacro-end-or-call-macro  &  <f4>               \\
\end{tabular}
\end{center}
\section{Terminal}
\label{sec-12}


\begin{center}
\begin{tabular}{ll}
 \textbf{Action}  &  \textbf{Shortcut}  \\
\hline
 eshell           &  C-t                \\
 new eshell       &  C-x M              \\
\end{tabular}
\end{center}
\section{Org}
\label{sec-13}

\begin{longtable}{ll}

 \textbf{Action}                               &  \textbf{Shortcut}    \\
\hline
\endhead
\hline\multicolumn{2}{r}{Continued on next page}\
\endfoot
\endlastfoot
 \textbf{Headings}                             &                        \\
 rotate entire buffer visbiliy                 &  S-TAB                 \\
 next/previous heading                         &  C-c C-n/p             \\
 next/previous heading, same level             &  C-c C-f/b             \\
 backward to higher level heading              &  C-c C-u               \\
 jump to another place in document             &  C-c C-j               \\
 previous/next plain list item                 &  S-up/down             \\
 insert new heading/item at current level      &  M-RET                 \\
 insert new heading after subtree              &  C-RET                 \\
 insert new TODO entry/checkbox item           &  M-S-RET               \\
 insert TODO entry/ckbx after subtree          &  C-S-RET               \\
 turn (head)line into item, cycle item type    &  C-c -                 \\
 turn item/line into headline                  &  C-c *                 \\
 promote/demote heading                        &  M-left/right          \\
 promote/demote current subtree                &  M-s-left/right        \\
 move subtree/list item up/down                &  M-s-up/down           \\
 clone a subtree                               &  C-c C-x c             \\
 copy visible text                             &  C-c C-x v             \\
 kill/copy subtree                             &  C-c C-x C-w/M-w       \\
 yank subtree                                  &  C-c C-x C-y or C-y    \\
 narrow buffer to subtree / widen              &  C-x n s/w             \\
\hline
 \textbf{Tables}                               &                        \\
 convert region to table                       &  C-c vertical line     \\
 org-table-insert-line                         &  C-c -                 \\
 re-align the table without moving the cursor  &  C-c C-c               \\
 re-align the table, move to next field        &  TAB                   \\
 move to previous field                        &  S-TAB                 \\
 re-align the table, move to next row          &  RET                   \\
 move to beginning/end of field                &  M-a/e                 \\
 move the current column left                  &  M-left/right          \\
 kill the current column                       &  M-S-left              \\
 insert new column to left of cursor position  &  M-S-right             \\
 move the current row up/down                  &  M-up/down             \\
 kill the current row or horizontal line       &  M-S-up                \\
 insert new row above the current row          &  M-S-down              \\
 insert hline below (C-u : above) current row  &  C-c -                 \\
 insert hline and move to line below it        &  C-c RET               \\
 export as tab-separated file                  &  M-x org-table-export  \\
 import tab-separated file                     &  M-x org-table-import  \\
 sum numbers in current column/rectangle       &  C-c +                 \\
\hline
 \textbf{Links, Footnotes and Images}          &                        \\
 org-mac-link-grabber                          &  C-c g                 \\
 org-insert-link                               &  C-c C-l               \\
 insert a link (TAB completes stored links)    &  C-c C-l               \\
 insert file link with file name completion    &  C-u C-c C-l           \\
 edit (also hidden part of) link at point      &  C-c C-l               \\
 open file links in emacs                      &  C-c C-o               \\
 \ldots{}force open in emacs/other window      &  C-u C-c C-o           \\
 find next link                                &  C-c C-x C-n           \\
 find previous link                            &  C-c C-x C-p           \\
 toggle inline display of linked images        &  C-c C-x C-v           \\
 org-footnote-action                           &  C-c C-x f             \\
\hline
 \textbf{Code and \LaTeX{}}                    &                        \\
 org-edit-src-code                             &  C-c ü                 \\
 org-edit-src-exit                             &  C-c ä                 \\
 org-pretty-entities                           &  C-c C-x \\            \\
 insert template of export options             &  C-c C-e t             \\
 org-cdlatex-mode                              &  C-c ö                 \\
 preview \LaTeX{} fragment                     &  C-c C-x C-l           \\
 expand abbreviation (cdlatex-mode)            &  TAB                   \\
 insert/modify math symbol (cdlatex-mode)      &  ` / `                 \\
 execute code block at point                   &  C-c C-c               \\
 open results of code block at point           &  C-c C-o               \\
 check code block at point for errors          &  C-c C-v c             \\
 insert a header argument with completion      &  C-c C-v j             \\
 view expanded body of code block at point     &  C-c C-v v             \\
 view information about code block at point    &  C-c C-v I             \\
 go to named code block                        &  C-c C-v g             \\
 go to named result                            &  C-c C-v r             \\
 go to the head of the current code block      &  C-c C-v u             \\
 go to the next code block                     &  C-c C-v n             \\
 go to the previous code block                 &  C-c C-v p             \\
 execute all code blocks in current buffer     &  C-c C-v b             \\
 execute all code blocks in current subtree    &  C-c C-v s             \\
 tangle code blocks in current file            &  C-c C-v t             \\
\hline
 \textbf{Items and Checkboxes}                 &                        \\
 rotate the state of the current item          &  C-c C-t               \\
 select next/previous state                    &  S-left/right          \\
 select next/previous set                      &  C-S-left/right        \\
 toggle ORDERED property                       &  C-c C-x o             \\
 insert new checkbox item in plain list        &  M-S-RET               \\
 toggle checkbox at point                      &  C-c C-c               \\
\end{longtable}
\section{\LaTeX}
\label{sec-14}


\begin{center}
\begin{tabular}{ll}
 \textbf{Action}              &  \textbf{Shortcut}  \\
\hline
 \LaTeX{}-math-abbrev-prefix  &  C-c m              \\
 \TeX{}-texify                &  C-c C-a            \\
\end{tabular}
\end{center}
\section{Ref\TeX}
\label{sec-15}


\begin{center}
\begin{tabular}{ll}
 \textbf{Action}  &  \textbf{Shortcut}  \\
\hline
 citation         &  C-c (              \\
 reference        &  C-c )              \\
 label            &  C-c l              \\
\end{tabular}
\end{center}
\section{Writing}
\label{sec-16}


\begin{center}
\begin{tabular}{ll}
 \textbf{Action}  &  \textbf{Shortcut}  \\
\hline
 writegood-mode   &  C-c w              \\
\end{tabular}
\end{center}
\section{ESS}
\label{sec-17}


\begin{center}
\begin{tabular}{ll}
 \textbf{Action}      &  \textbf{Shortcut}  \\
\hline
 ess-tracebug-prefix  &  M-p                \\
 ess-bp-set           &  F5                 \\
 ess-bp-kill          &  F6                 \\
\end{tabular}
\end{center}
\section{Python}
\label{sec-18}


\begin{center}
\begin{tabular}{ll}
 \textbf{Action}  &  \textbf{Shortcut}  \\
\hline
\end{tabular}
\end{center}
\section{Haskell}
\label{sec-19}


\begin{center}
\begin{tabular}{ll}
 \textbf{Action}       &  \textbf{Shortcut}  \\
\hline
 load file in ghci     &  C-c C-l            \\
 \textbf{Action}       &  \textbf{Shortcut}  \\
\hline
 inferior sbt session  &  C-c C-v s          \\
\end{tabular}
\end{center}

\end{document}
